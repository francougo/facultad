\documentclass[11pt, letterpaper]{article}
\usepackage[utf8]{inputenc}
\usepackage{amssymb}
\usepackage{graphicx}
\usepackage[T1]{fontenc}
\usepackage{charter}
\usepackage[a4paper, total={6in, 8in}]{geometry}
\graphicspath{ {/home/franco/Facultad/} }

\newcommand\vspt{15pt}
\newcommand\vsp{10pt}


\newcommand\tpn{4}
\newcommand\tptitle{Software ELQ}
\newcommand\tpsubj{Fisica II}

\begin{document}
\thispagestyle{empty}

\begin{center}\Huge{Trabajo Práctico \tpn: \tptitle\\
\vspace{\vspt}
\tpsubj}\end{center}

\vspace{\vspt}
\includegraphics[width=\linewidth]{utnlogo}

\vspace{\vspt}
\begin{center}\Huge{Franco Cosolito}\end{center}

\vspace{\vspt}
\begin{center}\Large{ISI C}\end{center}
\pagebreak

\setcounter{page}{1}
\begin{center}\textbf{Objetivos}\end{center}
    Estudio del campo y potencial eléctrico de un hilo recto, de 
    longitud finita y no conductor, ubicado en el vacío, en 
    ausencia de materiales conductores y dieléctricos, con
    una densidad lineal homogénea de cargas mediante el uso de
    software de simulación por ordenador.\\
\vspace{\vsp}

\pagebreak
\begin{center}\textbf{Desarrollo}\end{center}
    \textbf{1.} La aproximación del cálculo del campo eléctrico en un punto, en el 
    caso de N cargas puntuales se calcula según la ecuación 
    $\vec{E} = \frac{\vec{F_e}}{q_0} = \frac{q}{4 \pi \epsilon_0 r^2}$
    para cada carga puntual, por lo que el campo resultante de las N
    cargas se puede expresar segun sus componentes:\\
    En $x$: $$ \Sigma  E_i cos(\theta_i) $$
    En $y$: $$ \Sigma  E_i sen(\theta_i) $$
    Donde $\theta$ es el ángulo que forma la recta desde el punto a la carga 
    puntual con la horizontal.\\
    Al analizar los resultados obtenidos del software, se comete menores errores
    a medida que la carga se distribuye en mayor número de cargas puntuales, ya
    que los resultados se acercan más a el caso de infinitos puntos de carga.\\
    \textbf{2.} En todos los puntos con $x=0.5$, es decir alineados con el centro del
    cuerpo que contiene las cargas, el campo eléctrico es vertical. Las componentes
    $x$ de todas las cargas puntuales son iguales y de signo opuesto a las 
    componentes de otras cargas ubicadas a la misma distancia del centro del
    cuerpo por lo que se compensan.\\
    \textbf{3.} Para demostrar como el campo eléctrico varía respecto de la distancia a
    la linea de carga, se toman puntos en la linea del centro $x=0.5$ para 
    compensar las componentes en $x$. Luego al calcular el campo en distintos
    puntos de esta linea, se observa que el campo disminuye segun se toman puntos
    mas alejados de la linea de carga. Si se toman valores cercanos a la linea
    de carga el campo es proporcional a la inversa de la distancia a las cargas,
    mientras que si se toman valores mas alejados el campo es proporcional a la 
    inversa del cuadrado de la distancia.\\
    Para demostrarlo, ya que el sotware provee el valor del campo en $y$ y se 
    conoce la ecuacion del campo donde $\frac{q}{4\pi \epsilon_0}$ es constante,
    se puede plantear que $E.r=cte$. Entonces, para los valores de distancia
    $0.01m$ y $0.02m$ se conocen los valores del campo $1799640 N/C$ y $899280 N/C$
    respectivamente. Si se multiplican las distancias por los campos calculados,
    se obtienen los valores 17996.4 y 17985.6 respectivamente, valores 
    semejantes se hacen más cercanos entre sí segun se toman distancias menores
    a la linea de carga, ya que son la constante de la proporcion entre el
    campo y la distancia. Si se toman valores alejados de la linea, como $20m$
    y $25m$ se obtienen valores del campo eléctrico $22.5N/C$ y $14.4N/C$, 
    multiplicando los resultados por el cuadrado de sus distancias resulta
    ahora en la constante de proporción 9000. Lo que demuestra que a mayores
    distancias de la linea de carga el campo es proporcional a la inversa del
    cuadrado de la distancia. Además, si se divide a la constante por la carga
    utilizada en el software ($10^{-6} C$) se obtiene lo siguiente:
    $$ \frac{E.r^2}{q}= \frac{1}{4\pi \epsilon_0} = K = 9\times 10^9 $$
    Por lo que cumple la ley de Coulomb para cargas puntuales.\\
    En el caso de grandes distancias, la distancia es mucho mayor a las 
    dimensiones del cuerpo, de la linea, por lo que se desprecia su forma
    y se comporta como una carga puntual, pero para distancias pequeñas
    debe calcularse la contribución al campo de cada parte de la linea.\\
    Ya que el cuerpo es no conductor la carga está distribuida por igual
    a lo largo de la linea, es decir, en cualquier tramo la densidad de
    carga es igual. Entonces, si se calcula por integral el campo a una
    distancia pequeña $d$: $E = \frac{\lambda}{2 \pi \epsilon_0 d}$
    donde $\lambda$ es la densidad de carga. Como se expresa en la ecuación,
    el campo es proporcional a la inversa de la distancia y si se calcula
    $Ed = cte = 17983.6$ que es aproximadamente la constante de
    proporción encontrada para los valores $0.01$ y $0.02$.\\
    \textbf{4.} En las cercanías al centro de la linea las lineas de campo son 
    perpendiculares a la linea de carga debido a que las contribuciones
    horizontales de todas las cargas se compensan con las correspondientes
    ubicadas del otro lado del centro de la linea, si se calculan lineas
    de campo en posiciones cercanas a los extremos estas van dirigidas en la
    misma dirección que el extremo al cual estan más próximas. Esto se debe
    a que hay mayor cantidad de carga ubicada del lado opuesto al extremo
    en el cual se evalúa el campo, por lo que las contribuciones 
    horizontales no son compensadas completamente, es decir el campo se desvía
    hacia el extremo.\\
    \textbf{5.} No existen puntos donde se puedan anular o compensar las contribuciones
    de distintos elementos de carga al potencial ya que este es una magnitud 
    escalar, si se toman N cargas puntuales en la linea, cada una suma
    al potencial según la distancia al punto en el que se lo evalúa:
    $$ U = \frac{q_0}{4\pi \epsilon_0}(q_1/r_1 + q_2/r_2 + ... + q_N/r_N) $$
    \textbf{6.} Para distancias pequeñas con respecto a la linea de carga, las curvas
    equipotenciales tienen forma de elipse con eje mayor horizontal, en el sentido
    de la linea. A medida que se evaluan las curvas en distancias mayores las
    elipses se asemejan cada vez mas a una circunferencia. En cuanto al campo
    eléctrico, las lineas del campo son siempre perpendiculares a las curvas
    equipotenciales. La razón por la cual las curvas tienden a formar circunferencias
    o, en el espacio, esferas cuando la distancia es muy grande, es que las 
    dimensiones, la forma del cuerpo con las cargas se vuelven despreciables 
    y el conjunto de cargas se comporta como una carga puntual.\\
\pagebreak
\begin{center}\textbf{Conclusión}\end{center}
    El uso del software para resolver las consignas del trabajo práctico
    permitió analizar de forma práctica el comportamiento del campo 
    y el potencial eléctrico correspondientes a una linea
    de carga en distintos puntos y estudiar la relación entre ambos.
    Además de permitir observar de forma gráfica los cambios en 
    estas magnitudes con diferentes posiciones y distancias con
    respecto de las cargas.
\end{document}
