\documentclass[11pt, letterpaper]{article}
\usepackage[utf8]{inputenc}
\usepackage{amssymb}
\usepackage{amsmath}
\usepackage{graphicx}
\usepackage[T1]{fontenc}
\usepackage{charter}
\usepackage[a4paper, total={6in, 8in}]{geometry}
\graphicspath{ {/home/franco/Facultad/} }

\newcommand\vspt{15pt}
\newcommand\vsp{10pt}


\newcommand\tpn{5}
\newcommand\tptitle{Estudio de un circuito RC}
\newcommand\tpsubj{Física II}

\begin{document}
\thispagestyle{empty}

\begin{center}\Huge{Trabajo Práctico \tpn: \tptitle\\
\vspace{\vspt}
\tpsubj}\end{center}

\vspace{\vspt}
\includegraphics[width=\linewidth]{utnlogo}

\vspace{\vspt}
\begin{center}\Huge{Franco Cosolito}\end{center}

\vspace{\vspt}
\begin{center}\Large{ISI C}\end{center}
\pagebreak

\setcounter{page}{1}

\section{Marco teórico}
En este trabajo práctico se trabaja con un circuito RC para estudiar la carga
y descarga de su capacitor. Por lo tanto es importante tener en cuenta las
reglas de Kirchhoff y las ecuaciones que describen a la carga, la corriente
y el potencial eléctrico en función del tiempo para el circuito en carga y
en descarga.
\subsection{Reglas de Kirchhoff}
\begin{itemize}
    \item $\sum I = 0$ : Para cualquier unior de tres o mas conductores la suma
       de todas las corrientes es cero, es decir, la misma corriente que ingresa
       un conductor sale por los demás.
    \item $\sum V = 0$ : Si se recorre cualquier espira volviendo al punto inicial
        del circuito la suma de las FEM y caidas de tensión debe ser cero.
\end{itemize}
\subsection{Carga, corriente y potencial como funciones del tiempo}
\begin{itemize}
    \item Para un circuito en carga, es decir con una fuente que aporte una FEM:
        $$q=Q_f(1-e^{-t/RC})$$ 
        $$i=I_0e^{-t/RC}$$ 
        $$v_R=\mathcal{E} e^{-t/RC}$$
        $$v_C=\mathcal{E} (1-e^{-t/RC})$$
    \item Para un circuito en descarga:
        $$q=Q_0e^{-t/RC}$$ 
        $$i=I_0e^{-t/RC}$$ 
        $$v_R=v_C=\mathcal{E} e^{-t/RC}$$ 
\end{itemize}
        


\section{Materiales}
    materiales

\pagebreak
\section{Procedimiento}
Para medir la corriente y el potencial electrico primero se construyó un circuito
RC, este circuito consta de una fuente que aporta una FEM $\mathcal{E}$ seguido por
un capacitor con capacitancia $C=4700\times10^{-6}F$ y 
una resistencia $R=10860\;\Omega$, ubicados en serie en el circuito
inicialmente abierto. Luego, para efectuar las mediciones, se colocaron dos multímetros,
uno como voltímetro y el otro como amperímetro, en paralelo y en serie respectivamente.\\
Luego se configuró la fuente para que brinde una FEM de 15 Volts.
Con el circuito ya configurado se cerró el circuito permitiendo que circule corriente,
esto provocó que el capacitor se empiece a cargar, aumentando su diferencia de potencial
y disminuyendo la corriente del circuito. Se mantuvo el circuito en esta configuracion por
la cantidad de tiempo $\tau = RC = 51\;s$ para la cual el capacitor adquiere un potencial
de $\mathcal{E}(1-1/e) = 0.63\mathcal{E}$. Luego (con una carga diferente en el capacitor)
se quitó la fuente y se volvió a cerrar el circuito, permitiendo nuevamente que circule
corriente pero en este caso en sentido opuesto al anterior, ya que la corriente se debe
a la diferencia de potencial del capacitor unicamente ya que no hay mas FEM de la fuente.
Entonces la corriente parte de un valor máximo y disminuye a medida que pasa el tiempo
y el capacitor se descarga, disminuyendo tambien su voltaje.
\subsection{Carga del capacitor}
\begin{tabular}{ || c | c || }
    \hline
    $I$ & $V_c$\\
    \hline\hline
    1,344 & 0\\
    \hline
\end{tabular}
\pagebreak
\section{Conclusión}
La experiencia realizada en el trabajo práctico permite y calcular
por el metodo de la mezcla el valor del calor específico de una sustancia
midiendo y analizando los cambios de temperatura entre las partes de un sistema
debidos a la transferencia de calor.
\end{document}
