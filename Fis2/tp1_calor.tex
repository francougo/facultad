\documentclass[11pt, letterpaper]{article}
\usepackage[utf8]{inputenc}
\usepackage{amssymb}
\usepackage{graphicx}
\usepackage[T1]{fontenc}
\usepackage{charter}
\usepackage[a4paper, total={6in, 8in}]{geometry}
\graphicspath{ {/home/franco/Facultad/Fis2/} }

\newcommand\vspt{15pt}
\newcommand\vsp{10pt}
\newcommand\celc{^\circ C}


\newcommand\tpn{1}
\newcommand\tptitle{Medición de calor específico}
\newcommand\tpsubj{Fisica II}

\begin{document}
\thispagestyle{empty}

\begin{center}\Huge{Trabajo Práctico \tpn: \tptitle\\
\vspace{\vspt}
\tpsubj}\end{center}

\vspace{\vspt}
\includegraphics[width=\linewidth]{utnlogo}

\vspace{\vspt}
\begin{center}\Huge{Franco Cosolito}\end{center}

\vspace{\vspt}
\begin{center}\Large{ISI C}\end{center}
\pagebreak

\setcounter{page}{1}
\begin{center}\textbf{Objetivos}\end{center}
Medir el calor específico del alcohol etílico
 utilizando el método de las mezclas.\\
\vspace{\vsp}

\begin{center}\textbf{Marco teórico}\end{center}
\textbf{Equilibrio Térmico:} Dos cuerpos estan en equilibrio térmico 
si y solo si tienen la misma temperatura. Cuando esto ocurre
no se transfiere mas calor entre ellos.\\
\textbf{Temperatura:} Con la definición de equilibrio térmico
también se puede definir a la temperatura, es la propiedad de
los cuerpos que determina si estan o no en equilibrio 
térmico.\\
\textbf{Capacidad Calorífica:} Es la energía necesaria para elevar 
la temperatura de una masa, relaciona la energia termica
o calor y la variación de temperatura de un cuerpo:
$ Q = C \Delta T $\\
\textbf{Calor Específico:} Es la capacidad calorífica de una 
sustancia por unidad de masa: 
$ c = C/m = Q/{m \Delta T } $
\vspace{\vsp}

\begin{center}\textbf{Materiales}\end{center}
Para el trabajo práctico se emplearon los siguientes materiales:\\
\begin{itemize}
        \item Termo o calorímetro, para lograr condiciones cercanas a un sistema
                adiabático.
        \item Matraz aforado, para medir volumenes de agua.
        \item Balanza, para medir la masa del agua y el alcohol.
        \item Embudo.
        \item Termómetro, para medir las temperaturas.
\end{itemize}
                
\pagebreak
\begin{center}\textbf{Procedimiento}\end{center}
El objetivo del procedimiento es calcular el valor del calor 
específico del alcohol etílico. Para eso, el primer paso es
identificar que datos se deben medir.\\
Como el sistema que se va a emplear para realizar las mediciones
es adiabático, se debe cumplir que el calor cedido por cualquiera
de sus cuerpos o sustancias sea absorbido por otra parte del 
sistema, es decir $Q_{cedido} = Q_{absorbido}$.\\
Si se mezclan alcohol etílico y agua en el termo y 
utilizando la ecuacion del calor transferido en función de la 
masa, el calor específico y la variación de la temperatura:
$Q = m.c.\Delta T$, se debe cumplir que 
$$m_{ag}.c_{ag}.\Delta T_{ag} + m_{al}.c_{al}.\Delta T_{al} + Q_{termo} = 0$$
En principio se conocen el calor específico y la masa del agua
y se puede medir con un termómetro su temperatura (en equilibrio
térmico con el termo), pero se desconocen las sustancias que 
contiene el termo. Entonces primero debe calcularse de otra
forma que cantidad de calor representa $Q_{termo}$.
Para esto se plantea un procedimiento similar al descrito para
el caso del alcohol etílico, en el termo se mezclan dos volúmenes
de agua a diferente temperatura y se analizan las variaciones
para despejar la energía que absorbe el termo y poder calcular
su calor específico:\\
$$m_{ag1}.c_{ag1}.\Delta T_{ag1} + m_{ag2}.c_{ag2}.\Delta T_{ag2} + m_{ter} 
.c_{ter}.\Delta T_{ag1} = 0$$\\
En la ecuación planteada, es posible medir todas las variables 
de un experimento, exeptuando por el calor específico del termo
que luego se puede calcular.\\
Para representar el calor absorbido por el termo, se plantea una masa de 
agua $\pi$ tal que su capacidad calorífica sea igual a la del termo, es decir:
$$ C_{agua} = C_{termo}$$
$$c_{agua}.\pi = c_{termo}.m_{termo}$$
$$\pi = \frac{c_{termo}.m_{termo}}{c_{agua}}$$
Entonces:
$$m_{ag1}.c_{ag}.\Delta T_{ag1} + m_{ag2}.c_{ag}.\Delta T_{ag2} + \pi 
.c_{ag}.\Delta T_{ag1} = 0$$\\
Entonces el primer paso de la experiencia es pesar una cantidad inicial
de agua natural y agregarla al termo, esperar a que alcance el equilibrio térmico
con el recipiente y medir la temperatura con el termómetro.
En la experiencia, la masa de agua fue de 200g y la temperatura en equilibrio
fue de $24.1 \celc$.\\
Luego se mide la temperatura de una masa de 113.5g de agua caliente y se la agrega al 
termo mezclándola con el agua fría. Cuando esta mezcla y el termo lleguen
al equilibrio térmico, se mide la temperatura, que es el valor final de las 
variaciones $\Delta T$.
En la experiencia el agua caliente tenía una temperatura de $70 \celc$ y la 
temperatura final fue $39.9 \celc$. Además, se conoce el valor del calor
específico del agua: $4,186$ $J/{g\celc}$.\\
Entonces, despejando $\pi$ y sustituyendo los valores:\\
$$\pi = -\frac{m_{ag1}.\Delta T_{ag1} + m_{ag2}.\Delta T_{ag2}}{\Delta T_{ag1}}$$
\vspace{\vsp}
$$\pi = -\frac{200g.(39.9\celc - 24.1\celc) + 113.5g.(39.9\celc - 70\celc)}{(39.9\celc - 24.1\celc)}$$
\vspace{\vsp}
$$\pi = 16.22g$$
Con el valor $\pi$ se puede plantear la primera ecuación de manera que
$$Q_{termo} = \pi.c_{ag}.\Delta T$$
\vspace{\vsp}
$$m_{ag}.c_{ag}.\Delta T_{ag} + m_{al}.c_{al}.\Delta T_{al} + \pi.c_{ag}.\Delta T_{ag} = 0$$
\vspace{\vsp}
$$c_{al} = -\frac{m_{ag}.c_{ag}.\Delta T_{ag} + \pi.c_{ag}.\Delta T_{al}}{m_{al}.\Delta T_{al}}$$
Luego el procedimiento consiste en agregar al calorímetro 200g de alcohol etílico
y esperar a que alcance el equilibrio térmico con el recipiente, en la experiencia
la temperatura de equilibrio fue de $27\celc$. Con el sistema en equilibrio se
agregan $99.8g$ de agua caliente, en la experiencia a $72\celc$, que alcanza 
el equilibrio con la mezcla en $45.9\celc$. Entonces, luego de tomar las mediciones
con el termómetro en cada paso se puede resolver la ecuación planteada, considerando
también que $\Delta T_{\pi} = \Delta T_{al}$ ya que el alcohol inicialmente se 
encuentra en equilibrio térmico con el calorímetro y luego alcanzan el equilibrio
nuevamente con el agua caliente.
$$c_{al} = -\frac{99.8g.4,186J/g.(45.9\celc - 72\celc) + 16.22g.4,186J/g.(45.9\celc - 27\celc)}
{200g.(45.9\celc - 27\celc)}$$
\vspace{\vsp}
$$c_{al} = 2.54J/g$$
El valor para el calor específico del alcohol etílico obtenido de la experiencia
es $2.54$ $J/g$.
\pagebreak
\begin{center}\textbf{Conclusión}\end{center}
La experiencia realizada en el trabajo práctico permite y calcular
por el metodo de la mezcla el valor del calor específico de una sustancia
midiendo y analizando los cambios de temperatura entre las partes de un sistema
debidos a la transferencia de calor.
\end{document}
