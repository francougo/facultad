\documentclass[11pt, letterpaper]{article}
\usepackage[utf8]{inputenc}
\usepackage{amssymb}
\usepackage{graphicx}
\usepackage[T1]{fontenc}
\usepackage{charter}
\usepackage[a4paper, total={6in, 8in}]{geometry}
\graphicspath{ {/home/franco/Facultad/} }

\newcommand\vspt{15pt}
\newcommand\vsp{10pt}


\newcommand\tpn{3}
\newcommand\tptitle{Experiencias electroestáticas II}
\newcommand\tpsubj{Fisica II}

\begin{document}
\thispagestyle{empty}

\begin{center}\Huge{Trabajo Práctico \tpn: \tptitle\\
\vspace{\vspt}
\tpsubj}\end{center}

\vspace{\vspt}
\includegraphics[width=\linewidth]{utnlogo}

\vspace{\vspt}
\begin{center}\Huge{Franco Cosolito}\end{center}

\vspace{\vspt}
\begin{center}\Large{ISI C}\end{center}
\pagebreak

\setcounter{page}{1}
\begin{center}\textbf{Objetivos}\end{center}
    Comprobar la existencia del campo eléctrico y lograr la comprensión de sus 
    propiedades mediante experimentación.
\vspace{\vsp}

\begin{center}\textbf{Marco teórico}\end{center}
    \textbf{Jaula de Faraday:} Consiste en un fenómeno que ocurre en el interior
    de un cuerpo conductor con una cavidad, es decir en un recipiente cerrado conductor.
    Al exponer a un campo eléctrico externo, en el conductor se reordenan las cargas
    provocando que las positivas se desplacen hacia el sentido positivo del campo
    y las negativas en sentido opuesto, generando así un campo de igual magnitud
    y con sentido opuesto al externo en todo el conductor. De esta manera el campo
    resultante en el interior, en la cavidad, es nulo.\\
    \textbf{Generador de Van de Graaff:} Es un dispositivo que tiene un conductor
    hueco aislado que se carga a través de una cinta de material aislante.
    La cinta, posicionada de forma vertical, se mueve por la acción de un rodillo
    ubicado en el extremo inferior, opuesto al cuerpo conductor. La fricción con
    el rodillo carga a este y a la cinta por el efecto triboeléctrico.
    Además, se genera una alta concentración de carga en el rodillo, provocando
    que cargas opuestas a este (y de igual signo a las presentes en la cinta)
    se muevan desde un 'peine' conductor conectado a tierra, por el aire ionizado,
    hacia el rodillo, pero siendo captadas por la cinta.\\
    Luego las cargas de la cinta son transportadas y depositadas en el conductor
    que tambien posee un peine cercano a la cinta que permite la ionizacion del
    aire.\\
    El resultado de este proceso es que el conductor aislado adquiere una alta
    concentración de cargas que se ubican en su superficie.\\
    \textbf{Ionización del aire:} Ante la presencia de un campo eléctrico
    algunas de las cargas de un conductor cargado pueden desplazarse 
    a moléculas del aire, generando iones. Estos iones son partículas
    cargadas y libres de moverse por el aire, por lo que las fuerzas 
    eléctricas debidas al campo las aceleran. De esta forma el aire
    se vuelve conductor de cargas.\\
    \textbf{Efecto punta:} En materiales conductores, las cargas tienden
    a acumularse en las puntas de la superficie, resultando en una mayor
    concentración de cargas y por lo tanto un mayor campo eléctrico.\\
\begin{center}\textbf{Materiales}\end{center}
    \begin{itemize}
        \item Barra de material dieléctrico y paño de lana
        \item Electroscopio
        \item Jaula metálica
        \item Generador de Van de Graaff
        \item Campanario
        \item Pendulo con resorte
        \item Helice eléctrica
    \end{itemize}
\pagebreak
\begin{center}\textbf{Procedimiento}\end{center}
    \textbf{Jaula de Faraday}\\
        Para demostrar este fenómeno se frota la barra de material 
        dieléctrico con el paño de lana para cargarla y se la
        acerca al electroscopio. Lo que se observa es que sus 
        puntas se separan y elevan, es decir se reordenan sus 
        cargas debido al campo eléctrico generado por la barra
        cargada. Luego se encierra al electroscopio en una jaula
        metálica, conductora, y se vuelve a acercar la barra
        cargada. En este caso no se observa ningun cambio en el
        electroscopio, probando que no hay ningun campo eléctrico
        en el interior de la jaula metálica.\\
    \textbf{Generador de Van de Graaff}\\
        Luego de encender el generador y por lo tanto de que el 
        conductor aislado adquiera cargas, se coloca un plumerillo
        en contacto con el conductor. Lo que se observa es que
        sus tiras se estiran y se elevan, separandose entre sí.
        Esto se debe a que al adquirir parte de las cargas del 
        generador las cargas presentes en las tiras se repelen
        con las demas cargas ubicadas en las demás tiras.\\
        Para la segunda experiencia se emplea un campanario
        que tiene dos campanas: una se sostiene sobre un material 
        aislante y la otra está conectada a tierra a través de un
        conductor, entre ambas un cuerpo pequeño conductor cuelga
        de un hilo aislante. Al conectar la campana aislada al 
        generador el cuerpo oscila golpeando ambas campanas.
        Esta oscilación se debe a que inicialmente la fuerza 
        eléctrica ejercida por la campana cargada reordena las
        cargas en el cuerpo ubicando las cargas opuestas más 
        cerca de la campana, de esta forma las fuerzas de atracción
        se vuelven mayores a las de repulsión y el cuerpo se 
        desplaza hasta entrar en contacto con la campana cargada.
        El contacto provoca que el cuerpo se cargue con el mismo
        signo que la campana cargada, además las cargas de la 
        campana y del cuerpo inducen una carga de signo opuesto en 
        la campana conectada a tierra, por lo que el cuerpo es
        repelido por la campana aislada y atraido por la otra.
        Luego de entrar en contacto con la segunda campana
        el cuerpo se descarga y vuelve a repetirse cada paso.\\
        Para la tercera experiencia se emplea un molinillo 
        metálico que se ubica de manera que esté en contacto
        con el conductor del generador y pueda girar de forma
        horizontal. Lo que ocurre es que el molinillo gira
        debido a que en sus puntas, por el efecto punta,
        la alta concentración de cargas provoca que se 
        ionice el aire. Luego, al existir fuerzas de repulsión
        entre las partículas ionizadas y las puntas del molinillo
        existe una torca asociada a cada una de las puntas en el
        mismo sentido de giro, lo que produce la aceleración del
        molinillo.\\
        
\pagebreak
\begin{center}\textbf{Conclusión}\end{center}
La experiencia realizada en el trabajo práctico permite comprobar
las propiedades del campo eléctrico,  de las fuerzas eléctricas y 
observar el fenómeno de la jaula de Faraday.
\end{document}
