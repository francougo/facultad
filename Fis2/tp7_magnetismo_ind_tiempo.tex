\documentclass[11pt, letterpaper]{article}
\usepackage[utf8]{inputenc}
\usepackage{amssymb}
\usepackage{amsmath}
\usepackage{graphicx}
\usepackage{subcaption}
\usepackage[T1]{fontenc}
\usepackage{charter}
\usepackage[a4paper, total={6in, 8in}]{geometry}
\graphicspath{ {/home/franco/Facultad/Fis2/} }

\newcommand\vspt{15pt}
\newcommand\vsp{10pt}
\renewcommand{\figurename}{Fig.}


\newcommand\tpn{7}
\newcommand\tptitle{Experiencia electromagnetismo independiente del tiempo}
\newcommand\tpsubj{Física II}

\begin{document}
\thispagestyle{empty}

\begin{center}\Huge{Trabajo Práctico \tpn: \tptitle\\
\vspace{\vspt}
\tpsubj}\end{center}

\vspace{\vspt}
\includegraphics[width=\linewidth]{utnlogo}

\vspace{\vspt}
\begin{center}\Huge{Franco Cosolito}\end{center}

\vspace{\vspt}
\begin{center}\Large{ISI C}\end{center}
\pagebreak

\setcounter{page}{1}
\section{Objetivos}
Observar e interpretar una serie de experiencias donde se ponen de 
manifiesto los principales fenómenos electromagnéticos. Orientado
hacia dos objetivos fundamentales: 
\begin{itemize}
    \item Campos magnéticos estacionarios originados por imanes
        permanentes y corrientes continuas, estudio del espectro
        de líneas de campo.
    \item Fuerzas magnéticas de interaccion: corriente - campo.
\end{itemize}

\section{Marco teórico}
El campo magnético puede ser generado de distintas fuentes: puede ser
producto del movimiento a velocidad constante de una partícula con carga,
puede deberse a una corriente que circule por un conductor o puede ser
causado por un imán permanente.\\
Los imanes poseen dos polos, norte y sur, que no pueden ser separados en
dos piezas y tienen la caracteristica de atraerse o repelerse con otros
imanes, los polos iguales se repelen y los diferentes se atraen.

\section{Materiales}
\begin{itemize}
    \item Brújula
    \item Imanes permanentes
    \item Limaduras de hierro
    \item Espiras conductoras
\end{itemize}

\pagebreak

\section{Procedimiento}
\subsection{Identificación de polos en imanes permanentes}
Primero se comprobó que las brújulas utilizadas funcionen correctamente,
todas las brújulas deben apuntar hacia el norte geográfico (a menos que
se les acerque un iman), ya que este es un polo sur magnético.
Luego, se acercó una brújula a uno de los imanes, esto provocó que la
aguja de la brújula se desvie de la dirección original para apuntar
a uno de los extremos del iman, de esta manera se pudo identificar el
polo sur. Después se movió la brujula a poca distancia del iman y se
observó que la aguja se ubicó de forma casi paralela al imán y luego,
cerca del otro extremo, apuntó en dirección contraria, permitiendo 
identificar el polo norte.
\subsection{Lineas de campo magnético en imanes permanentes}
Para observar las lineas de campo en un imán permanente se utilizó
una pieza de gel con limaduras de hierro suspendidas en el. Esta pieza
tiene, además, una cavidad en el centro donde se puede colocar un imán.
Estas limaduras se pueden mover libremente por la sustancia transparente
lo que permite que, al acercar un imán o colocarlo en la cavidad, se 
ubiquen de forma tangente a las lineas de campo magnético. Este es
el mismo comportamiento que presentó la brujula en el entorno del imán,
el polo norte de la brújula o de la limadura apunta hacia la dirección
del polo sur siguiendo la 'trayectoria' que marca una de las lineas de
campo.\\
\begin{figure}[h]
    \centering
    \begin{subfigure}{0.4\linewidth}
        \includegraphics[width=\linewidth]{limaduras1.jpg}
    \end{subfigure}
    \begin{subfigure}{0.4\linewidth}
        \includegraphics[width=\linewidth]{limaduras2.jpg}
    \end{subfigure}
    \caption{Limaduras de hierro sobre las lineas de campo}
\end{figure}
En las imagenes de la figura 1 se observan las limaduras sobre las 
lineas de campo, que se encuentran en gran concentración cerca de los
polos de los imanes y en menor cantidad a medida que se alejan.


\subsection{Campo magnético dentro de una espira con corriente}
Para observar la presencia de campo magnético cerca de un conductor por
el que circula corriente se utilizo un dispositivo con una espira conductora
de varias vueltas de cable. La experiencia consistió en conectar el dispositivo
a una fuente en paralelo con un amperímetro para comprobar que circule corriente
y luego colocar una brújula en el centro de la espira.\\
\begin{figure}[h]
    \centering
    \includegraphics[width=0.8\linewidth]{circuito.jpg}
    \caption{Espira conductora conectada a la fuente}
\end{figure}
Lo que se observó luego de colocar la brújula cerca del centro de la espira es
que se aguja se desvió de su orientación hacia el norte geográfico y se ubicó
de forma perpendicular a la espira. Esto se debe a que la corriente que circula
por la espira genera un campo magnético a su alrededor con la misma dirección.

\subsection{Fuerza magnética}
- conductor en espira con un eje que le permite girar, gira al 
haber un campo externo y circular corriente (ver videos)

\pagebreak
\section{Conclusión}
La experiencia permitió estudiar el funcionamiento de un circuito
RC y comprobar a traves del uso de instrumentos de medición los
comportamientos de sus variables como voltaje, corriente y carga
descritos por las funciones tanto en el regimen de carga como en
descarga y analizar la relación entre dichas variables y la 
constante de tiempo del circuito.
\end{document}
