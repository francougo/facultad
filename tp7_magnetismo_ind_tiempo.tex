\documentclass[11pt, letterpaper]{article}
\usepackage[utf8]{inputenc}
\usepackage{amssymb}
\usepackage{amsmath}
\usepackage{graphicx}
\usepackage[T1]{fontenc}
\usepackage{charter}
\usepackage[a4paper, total={6in, 8in}]{geometry}
\graphicspath{ {/home/franco/Facultad/} }

\newcommand\vspt{15pt}
\newcommand\vsp{10pt}
\renewcommand{\figurename}{Fig.}


\newcommand\tpn{7}
\newcommand\tptitle{Experiencia electromagnetismo independiente del tiempo}
\newcommand\tpsubj{Física II}

\begin{document}
\thispagestyle{empty}

\begin{center}\Huge{Trabajo Práctico \tpn: \tptitle\\
\vspace{\vspt}
\tpsubj}\end{center}

\vspace{\vspt}
\includegraphics[width=\linewidth]{utnlogo}

\vspace{\vspt}
\begin{center}\Huge{Franco Cosolito}\end{center}

\vspace{\vspt}
\begin{center}\Large{ISI C}\end{center}
\pagebreak

\setcounter{page}{1}
\section{Objetivos}
Observar e interpretar una serie de experiencias donde se ponen de 
manifiesto los principales fenómenos electromagnéticos. Orientado
hacia dos objetivos fundamentales: 
\begin{itemize}
    \item Campos magnéticos estacionarios originados por imanes
        permanentes y corrientes continuas, estudio del espectro
        de líneas de campo.
    \item Fuerzas magnéticas de interaccion: corriente - campo.
\end{itemize}

\section{Marco teórico}

\section{Materiales}
\begin{itemize}
    \item Brújula
    \item Imanes permanentes
    \item Limaduras de hierro
    \item Espiras conductoras
\end{itemize}

\pagebreak

\section{Procedimiento}
- buscar polos con brujulas en un iman permanente
- observar lineas de campo a partir de limaduras rodeando un iman
- lo mismo pero para una placa con un conductor en espira
- conductor en espira con un eje que le permite girar, gira al 
haber un campo externo y circular corriente (ver videos)

\pagebreak
\section{Conclusión}
La experiencia permitió estudiar el funcionamiento de un circuito
RC y comprobar a traves del uso de instrumentos de medición los
comportamientos de sus variables como voltaje, corriente y carga
descritos por las funciones tanto en el regimen de carga como en
descarga y analizar la relación entre dichas variables y la 
constante de tiempo del circuito.
\end{document}
